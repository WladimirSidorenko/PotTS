\documentclass[12pt,fleqn]{scrreprt}

%%%%%%%%%%%%%%%%%%%%%%%%%%%%%%%%%%%%%%%%%%%%%%%%%%%%%%%%%%%%%%%%%%
%% Libraries
\usepackage[top=2.5cm,bottom=2.5cm,left=2.5cm,right=2.5cm]{geometry}
\usepackage[latin1]{inputenc}
\usepackage{paperandpencil}
\usepackage{eurosym}
\usepackage{wasysym}   % smiley symbols
\usepackage{multicol}

%%%%%%%%%%%%%%%%%%%%%%%%%%%%%%%%%%%%%%%%%%%%%%%%%%%%%%%%%%%%%%%%%%
%% Title
\author{Wladimir Sidorenko}
\date{\today}
\title{Sentiment Annotation Experiment}

%%%%%%%%%%%%%%%%%%%%%%%%%%%%%%%%%%%%%%%%%%%%%%%%%%%%%%%%%%%%%%%%%%
%% Main
\begin{document}
\begin{center}
{\LARGE{\bfseries Sentiment Annotation Test}}
\end{center}

\begin{flushright}
Name: \underline{\hspace{6cm}}

Surname: \underline{\hspace{6cm}}

Date: \underline{\hspace{6cm}}
\end{flushright}

\section*{Definitions}
You might find the definitions below useful when making the test:

\textbf{Emo-expressions} (\textit{expressive subjective elements}) -
lexical items with polar evaluative sense, e.g. \textit{gut,
  schrecklich, kritisieren, zum Besten halten etc.};

\textbf{Diminishers} (\textit{down-toners}) - words or phrases which
decrease the intensity of an emo-expression term,
e.g. \textit{weniger, bisschen, kaum etc.}

\textbf{Intensifiers} - lexical elements which strengthen the polar
evaluative sense of an emo-expression, e.g. \textit{recht, super,
  au\ss{}erordentlich etc.}

\textbf{Negations} - language elements which reverse the polarity of
subjective meaning expressed by an ESE, e.g. \textit{nicht, kein,
  etc.}

\textbf{Sentiment} - minimal complete coherent syntactic or
discourse-level unit that expresses a polar evaluative opinion of a
person or organization about some particular subject, topic, or event,
e.g. \textit{Ich hasse diese Reform}, \textit{ein ausgezeichneter
  Film}, \textit{Meine Mutter ruft mich heulend an.  Man hat einen
  Argentinier zum Papst gew\"ahlt.};

\textbf{Source} - the immediate originator of a polar evaluative
opinion who either directly expresses her opinion or whose opinion is
being cited;

\textbf{Target} - subject or event which is being evaluated in a
sentiment.

\section*{Test}
\begin{enumerate}[I)]
  {\bfseries \item Write down the names of all markable types that are
    defined in our annotation scheme:}
  \begin{multicols}{2}
    \begin{enumerate}
    \item
    \item
    \item
    \item
    \item
    \item
    \item
    \end{enumerate}
  \end{multicols}

  {\bfseries\item Which tags from our annotation sheme would you assign
    to the words shown in boldface in the following examples (select
    none if you think that nothing applies):}
  \begin{enumerate}
  \item Mir \textbf{gefiel} das Spiel der deutschen Nationalelf gegen
    Brasilien.
    \begin{answersA}
    \item \ebox{diminisher}
    \item \ebox{emo-expression}
    \item \ebox{intensifier}
    \item \ebox{sentiment}
    \end{answersA}

  \item Ich mag ihn \textbf{sooooo} gern!!!
    \begin{answersA}
    \item \ebox{diminisher}
    \item \ebox{intensifier}
    \item \ebox{negation}
    \item \ebox{sentiment}
    \end{answersA}

  \item Derjenige, der morgen mit der Klasse nach Bad Liebenswerde f\"ahrt,
    \textbf{m\"oge} folgende Anweisung bitte lesen
    \begin{answersA}
    \item \ebox{emo-expression}
    \item \ebox{intensifier}
    \item \ebox{sentiment}
    \item \ebox{target}
    \end{answersA}

  \item Ich h\"atte \textbf{es} mir gern erspart aber es ging nicht anders.
    \begin{answersA}
    \item \ebox{emo-expression}
    \item \ebox{intensifier}
    \item \ebox{sentiment}
    \item \ebox{target}
    \end{answersA}

  \item Laut \textbf{FAZ} soll der fr\"uhere Bundespr\"asident Christian Wulff
    scharfe Kritik an Justiz und Medien wegen ihrer Rolle in der Aff\"are
    um ihn ge\"ubt haben.
    \begin{answersA}
    \item \ebox{emo-expression}
    \item \ebox{sentiment}
    \item \ebox{source}
    \item \ebox{target}
    \end{answersA}
  \end{enumerate}

  {\bfseries\item Please complete the following sentences:}
  \begin{enumerate}
  \item In order to decide, if a given expression forms a sentiment,
    we should check if it fulfills the following three criterie,
    namely if it is \underline{subjective}, \underline{\hspace{4cm}},
    and \underline{\hspace{4cm}}.

  \item The possible attributes of sentiments are
    \underline{polarity}, \underline{\hspace{4cm}}, and
    \underline{\hspace{4cm}}.

  \item If the target of a sentiment is expressed by multiple
    coordinatively conjoined noun phrases, we should annotate each
    such noun phrase \underline{\hspace{4cm}}.

  \item We should draw the \texttt{sentiment-ref} edge attribute for
    source when \underline{the \texttt{source} is located at
      the intersection of two different \texttt{sentiment}s} and
    when \underline{\hspace{8cm}}.

  \item When determining the polarity of \texttt{emo-expressions}, we
    should \underline{\hspace{4cm}} the effect of its contextual
    negations and modifiers.
  \end{enumerate}

  {\bfseries\item Please annotate sentiments (with square brackets
    []), targets (with parentheses ()), and emo-expressions (with an
    \underline{underline}) in the following examples:}
  \begin{enumerate}
  \item hab mir auf die Zunge gebissen :-(

  \item Heute ist Welttag gegen den L\"arm ich w\"unsche euch allen
    einen ruhigen Tag :)

  \item Wollte grade was zynisches \"uber Unterschiede zwischen
    \#Boston und \#Syrien schreiben. Finde mich selber zum kotzen. Mad
    World.

  \item Morgen mal FNP lesen \smiley{} "@JFroemmrich: Minister ohne
    Plan, ohne Konzept, nicht den Ansatz einer Idee \#Rhein
    http://t.co/yDCaoyY0XF

  \item Wahrscheinlich hat sich die \#CSU auch schon um Justin Bieber
    als Parteimitglied bem\"uht. \frownie{}

  \item \#Browsergames: \#Bundeskampf - \#K\"ampfe f\"ur dein
    Bundesland! Bei uns erbeutest du \#Bier und bek\"ampfst
    \#Mitspieler: http://t.co/W1DON5OYUB

  \item @iwrestledabeer @Zuckermensch @whoisxoxo naja, hab beide noch
    nicht gesehen. und wei\ss auch nicht wirklich, ob ich sie sehen
    will :(

  \item RT @MarcusBlumberg: Finde es aber bemerkenswert, dass die SPD
    mit aller Macht versucht, einen noch selbstzerst\"orerischen
    Wahlkampf zu machen...
  \end{enumerate}

  {\bfseries\item Please answer the following questions:}

  \question{What types of polarities do we distinguish for
    sentiments?}\opentwo

  \question{What are the main formal criteria for determining
    boundaries of sentiment, source, and target spans?}\opentwo

  \question{Should you always annotate insults as sentiments?}\opentwo

  \question{In which cases should you annotate as sentiments sentences
    which do not have any explicit evaluation expressions except for
    the smiley at the end?}\opentwo

  \question{Write down an example of a sentiment which does not have
    any explicit emo-expression but still expresses a polar evaluative
    judgement about a target.}\opentwo
\end{enumerate}
\end{document}
